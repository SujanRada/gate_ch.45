% \iffalse
\let\negmedspace\undefined
\let\negthickspace\undefined
\documentclass[journal,12pt,twocolumn]{IEEEtran}
\usepackage{cite}
\usepackage{amsmath,amssymb,amsfonts,amsthm}
\usepackage{algorithmic}
\usepackage{graphicx}
\usepackage{textcomp}
\usepackage{xcolor}
\usepackage{txfonts}
\usepackage{listings}
\usepackage{enumitem}
\usepackage{mathtools}
\usepackage{gensymb}
\usepackage{comment}
\usepackage[breaklinks=true]{hyperref}
\usepackage{tkz-euclide} 
\usepackage{listings}
\usepackage{gvv}                                        
\def\inputGnumericTable{}                                 
\usepackage[latin1]{inputenc}                                
\usepackage{color}                                            
\usepackage{array}                                            
\usepackage{longtable}                                       
\usepackage{calc}                                             
\usepackage{multirow}                                         
\usepackage{hhline}                                           
\usepackage{ifthen}                                           
\usepackage{lscape}

\newtheorem{theorem}{Theorem}[section]
\newtheorem{problem}{Problem}
\newtheorem{proposition}{Proposition}[section]
\newtheorem{lemma}{Lemma}[section]
\newtheorem{corollary}[theorem]{Corollary}
\newtheorem{example}{Example}[section]
\newtheorem{definition}[problem]{Definition}
\newcommand{\BEQA}{\begin{eqnarray}}
\newcommand{\EEQA}{\end{eqnarray}}
\newcommand{\define}{\stackrel{\triangle}{=}}
\theoremstyle{remark}
\newtheorem{rem}{Remark}
\begin{document}
\parindent 0px
\bibliographystyle{IEEEtran}
\title{GATE: CH - 45.2023}
\author{EE22BTECH11219 - Rada Sai Sujan$^{}$% <-this % stops a space
}
\maketitle
\newpage
\bigskip
\section*{Question}
Level \brak{h} in a steam boiler is controlled by manipulating the flow rate \brak{F} of the break-up(fresh) water using a proportional \brak{P} controller. The transfer function between the output and the manipulated input is   \\
$$ \frac{h\brak{s}}{F\brak{s}}=\frac{0.25\brak{1-s}}{s\brak{2s+1}} $$   \\
The measurement and the valve transfer functions are both equal to 1. A process engineer wants to tune the controller so that the closed loop response gives the decaying oscillations under the servo mode. Which one of the following is the CORRECT value of the controller gain to be used by the engineer? \\
\begin{enumerate}
    \item[(A)] $0.25$
    \item[(B)] $2$
    \item[(C)] $4$
    \item[(D)] $6$
\end{enumerate}
\solution
\begin{table}[ht]
    \centering
    \begin{tabular}{|p{2cm}|p{2cm}|p{3.8cm}|}
    \hline
    PARAMETER & VALUE  & DESCRIPTION \\ \hline
    $$G_c$$ & $$K_c$$ & Proportional controller's transfer function \\ \hline
    $$G_f$$ & $$1$$ & Valve transfer function \\ \hline
    $$G_p$$ & $$\frac{0.25\brak{1-s}}{s\brak{2s+1}}$$ & Process transfer function   \\ \hline
    $$G_M$$ & $$1$$ & Measurement transfer function \\ \hline 
    \end{tabular}

    \caption{PARAMETER TABLE 1}
    \label{tab:ch.45.1}
\end{table} \\
\begin{figure}[ht]
    \centering
    \includegraphics[width=\columnwidth]{figs/a.png}
    \caption{Block diagram}
    \label{fig:ch.45.1}
\end{figure}    
\begin{table}[ht]
    \centering
    \begin{tabular}{|p{2cm}|p{2cm}|p{4cm}|}
    \hline
    PARAMETER & VALUE & DESCRIPTION   \\ \hline
    $$X\brak{s}$$ & $$X\brak{s}$$ & Laplace transform of input signal  \\ \hline
    $$F\brak{s}$$ & $$X\brak{s}G_c$$ & Mnipulated input Laplace Transform \\ \hline
    $$h\brak{s}$$ & $$h\brak{s}$$ & Laplace transform of Output signal  \\ \hline
    $$G\brak{s}$$ & $$\frac{h\brak{s}G_c}{F\brak{s}}$$ & Open loop transfer function    \\ \hline
    $$H\brak{s}$$ & $$1$$ & Feedback  \\ \hline
    $$T\brak{s}$$ & $$\frac{h\brak{s}}{X\brak{s}}$$ & Transfer function of system   \\ \hline
\end{tabular}

    \caption{PARAMETER TABLE 2}
    \label{tab:ch.45.2}
\end{table} \\
Output signal transfer function of the above block diagram can be given by,
\begin{align}
     Y\brak{s} &= \frac{G\brak{s}}{1+G\brak{s}H\brak{s}}    \\
    \implies Y\brak{s} &= \frac{1-s}{2s^2 + \brak{1-0.25K_c}s + 0.25Kc}
\end{align}
From options, veryfying with the value of $K_c=2$, 
\begin{align}
    Y\brak{s} &= \frac{1}{2}\brak{\frac{1-s}{s^2+0.25s+0.25}}   \\
    \implies Y\brak{s} &= \frac{1}{2}\brak{\frac{1-s}{\brak{s+s_1}\brak{s+s_2}}} \label{eq:4}
\end{align}
Where,
\begin{align}
    s_1 &= \frac{1-\sqrt{15}j}{8}   \\
    s_2 &= \frac{1+\sqrt{15}j}{8}   
\end{align}
From \eqref{eq:4} we get,
\begin{align}
    Y\brak{s} &= \frac{1}{2\brak{s_2-s_1}}\brak{\frac{1+s_1}{s+s_1}-\frac{1+s_2}{s+s_2}}
\end{align}
Now taking the inverse laplace transform we have,
\begin{align}
    y\brak{t} &= \frac{1}{2\brak{s_2-s_1}} \left [\brak{1+s_1}e^{-s_1t}-\brak{1+s_2}e^{-s_2t} \right ]  \\
    \implies y\brak{t} &= \frac{e^{\frac{-t}{8}}}{2\sqrt{15}}\brak{9\sin \brak{\frac{\sqrt{15}t}{8}} - \sqrt{15}\cos \brak{\frac{\sqrt{15}t}{8}}}
\end{align}
Plotting the graph of the above function,   \\
\begin{figure}[ht]
    \centering
    \includegraphics[width=\columnwidth]{figs/b.png}
    \caption{y\brak{t} $vs$ t graph}
    \label{fig:ch.45.2}
\end{figure}    \\
We can observe decaying oscillations in the above graph,    \\
$$\therefore K_c=2$$
\end{document}
